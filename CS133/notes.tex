\documentclass[a4paper]{article}

\usepackage[utf8]{inputenc}
\usepackage[T1]{fontenc}
\usepackage{textcomp}
\usepackage{listings}
\usepackage{lmodern}
\usepackage{amsfonts}
\usepackage{titling}
\usepackage{lipsum}
\usepackage[left=1in, right=1in, bottom=1in, top=1in]{geometry}
\usepackage[backend=bibtex]{biblatex}
\usepackage{mathtools}
\usepackage{amsthm}
\usepackage{tcolorbox}
\usepackage{hyperref}
\usepackage{xcolor}
\usepackage{graphicx}
\usepackage{makeidx}
\usepackage{tikz}
\usepackage{cases}
\usepackage{apacite}
\usepackage{tkz-berge}
\usepackage{lastpage}
\usepackage{fancyhdr}
\pagestyle{fancy} 
\usepackage{url}
\usepackage{tgtermes}
\usepackage{sectsty}
\usepackage{subcaption}
\usepackage{setspace}
\usepackage{float}
\usepackage{amsmath, amssymb}
\rhead{Page~\thepage~of~\pageref{LastPage}.}
\cfoot{}

% figure support
\usepackage{import}
\usepackage{xifthen}
\pdfminorversion=7
\usepackage{pdfpages}
\usepackage{transparent}
\newcommand{\incfig}[2][1]{%
    \def\svgwidth{#1\columnwidth}
    \import{./figures/}{#2.pdf_tex}
}

%mathstyling
\theoremstyle{plain}
\newtheorem{thm}{Theorem}[section]
\newtheorem{lem}[thm]{Lemma}
\newtheorem{prop}[thm]{Proposition}
\newtheorem*{cor}{Corollary}

\theoremstyle{definition}
\newtheorem{defn}{Definition}[section]
\newtheorem{conj}{Conjecture}[section]
\newtheorem{exmp}{Example}[section]
\newtheorem{axiom}{Axiom}
\theoremstyle{remark}
\newtheorem*{rem}{Remark}
\newtheorem*{note}{Note}

\pdfsuppresswarningpagegroup=1

\begin{document}
\begin{titlepage}
\begin{center}
\large
University of Warwick \\
Department of Computer Science \\
\huge
\vspace{50mm}
\rule{\linewidth}{0.5pt} \\
CS133 \\
\vspace{5mm}
\Large
Professional Skills
\rule{\linewidth}{0.5pt}
\vspace{5mm}
\begin{figure}[H]
\centering
\includegraphics[width=0.4\textwidth]{crest_black.eps}
\end{figure}
\vspace{37mm}
Cem Yilmaz\\
\today
\end{center}
\end{titlepage}
\section{Linux}
\subsection{What is Linux?}
Linux is a free version of UNIX developed by a developer called Linus. They're compose of three components:
\begin{enumerate}
	\item Kernel;
	\item Shell: interface to kernel;
	\item Utilities: word editors etc.
\end{enumerate}
\begin{tcolorbox}[colback=black!3!white,colframe=black!60!white,title=\begin{defn}Shell \label{Shell}\end{defn}]
A shell can be seen as different things:
\begin{enumerate}
	\item Command interpreter;
	\item Friendly hooks to kernel;
	\item Programming language
\end{enumerate}
Some examples of shells include bash, Gnome, csh etc.
\end{tcolorbox}
\section{Linux Concepts}
\subsection{Files}
Files are used to store data, it can contain anything. Data are sequences of bytes. \\
These files can be classified as:
\begin{enumerate}
	\item Regular: text, binary code, HTML, Java, etc\ldots;
	\item Director;
	\item Device
\end{enumerate}
Each "file" has a unique ID, called inode. This is usually a number. All fires are in fact tree hierarchies. However, it is possible that directories can lead to the same files. This is when the also share inodes.
\subsection{Basic commands}
\subsubsection{Terminal Commands}

\begin{lstlisting}[language = Shell]
cd \\ current directory
pwd \\ print working director
mkdir \\ make a directory
rm \\ remove. Can also be used to do -d and -R to delete dir or recursively.
rmdir \\ same as rm -d but only works on empty dirs
cp \\ copy, can be also be used to do -R.
mv \\ move, moves a fily. Can be used to rename
ls \\ list, lists all files. You can use it to view vermissions with -l. Secret files with -a.
\end{lstlisting}
\subsubsection{Symbols}
\begin{lstlisting}[language = Shell]
. \\ current working directory
.. \\ parent directory
~ \\ home directory
/ \\ root
\end{lstlisting}
\subsection{Permissions}
Permissions are set up with binary code, specifically $111$.
\begin{tcolorbox}[colback=black!3!white,colframe=black!60!white,title=\begin{defn}Permissions \label{Permissions}\end{defn}]
A permission is defined by
\begin{align}
	\underbrace{-}_{\text{Type}}\underbrace{---}_{\text{User}}\underbrace{---}_{\text{Group}}\underbrace{---}_{\text{Other}}
\end{align}
where $---$ represents permissions, listing from $rwx$, where $r$ is read, $w$ is write, $x$ is execute. 
\end{tcolorbox}
The command $chmod$ is then used to change permissions. $u$ represents user, $g$ represents group, $o$ represents others and $a$ represents all. So, for example, one can run the command
\begin{lstlisting}[language = Shell]
chmod u+rwx,g-r,o+x ~/example.txt
\end{lstlisting}
And for this, would add $rwx$ for user, remove $r$ for group and add $x$ for others. Alternative ways of using chmod including converting $111,111,111$ system (without the $,$ ) to base $10$, e.g or instead of doing $\pm$, you can use $=$.\\
A tip is to use the man command, which lists a list of commands that you can use in the terminal.
\subsection{Process}
Every process has a special id number (PID, process ID) and are divided into streams. The process is entered through a standard input stream, and the result is then given by a standard output stream. The streams are:
\begin{enumerate}
	\item Standard Input stream - stdin, no.0\\
	\item Standard Output stream - stdout, no.1 \\
	\item Standard Error stream - stderr, no.2 \\
	\item Other streams and I/Os\ldots
\end{enumerate}
\subsection{Input}
Suppose we have a command called $myScript$ and the file $f$. We can redirect input to $myScript$ from $f$ using
\begin{lstlisting}[language = Shell]
myScript < f
myScript 0< f
\end{lstlisting}
You can create more streams by assigning more numbers to them like done above.
\subsection{Output}
Files are created implicitly as and when required.\\
We can overwrite the file
\begin{lstlisting}[language = Shell]
myScript > f
myScript 1> f
myScript 2> f
\end{lstlisting}
We can append to the end of the file
\begin{lstlisting}[language = Shell]
myScript >> f
myScript 1>> f
\end{lstlisting}
Current processes can be checked using the $ps$ command. You can also use the command $top$ to sort processes by system usage of CPU. You can use the command $kill$ to kill a process.
\subsection{Path}

\end{document}
