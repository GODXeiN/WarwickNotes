\documentclass[a4paper]{article}

\usepackage[utf8]{inputenc}
\usepackage[T1]{fontenc}
\usepackage{textcomp}
\usepackage{listings}
\usepackage{lmodern}
\usepackage{amsfonts}
\usepackage{titling}
\usepackage{lipsum}
\usepackage[left=1in, right=1in, bottom=1in, top=1in]{geometry}
\usepackage{amsthm}
\usepackage{tcolorbox}
\usepackage{hyperref}
\usepackage{xcolor}
\usepackage{graphicx}
\usepackage{makeidx}
\usepackage{tikz}
\usepackage{cases}
\usepackage{apacite}
\usepackage{tkz-berge}
\usepackage{url}
\usepackage{tgtermes}
\usepackage{sectsty}
\usepackage{subcaption}
\usepackage{setspace}
\usepackage{float}
\usepackage{amsmath, amssymb}


% figure support
\usepackage{import}
\usepackage{xifthen}
\pdfminorversion=7
\usepackage{pdfpages}
\usepackage{transparent}
\newcommand{\incfig}[2][1]{%
    \def\svgwidth{#1\columnwidth}
    \import{./figures/}{#2.pdf_tex}
}

%mathstyling
\theoremstyle{plain}
\newtheorem{thm}{Theorem}[section]
\newtheorem{lem}[thm]{Lemma}
\newtheorem{prop}[thm]{Proposition}
\newtheorem*{cor}{Corollary}

\theoremstyle{definition}
\newtheorem{defn}{Definition}[section]
\newtheorem{conj}{Conjecture}[section]
\newtheorem{exmp}{Example}[section]
\newtheorem{axiom}{Axiom}
\theoremstyle{remark}
\newtheorem*{rem}{Remark}
\newtheorem*{note}{Note}

\pdfsuppresswarningpagegroup=1

\begin{document}
	\begin{titlepage}
	\begin{center}
	\large
	University of Warwick \\
	Department of Computer Science \\
	\huge
	\vspace{50mm}
	\rule{\linewidth}{0.5pt} \\
	CS131 \\
	\vspace{5mm}
	\Large
	Mathematics for Computer Scientists II
	\rule{\linewidth}{0.5pt}
	\vspace{5mm}
	\begin{figure}[H]
	\centering
	\includegraphics[width=0.4\textwidth]{crest_black.eps}
	\end{figure}
	\vspace{37mm}
	Cem Yilmaz \\
	\today
	\end{center}
	\end{titlepage}
	\newpage
	\tableofcontents
	\newpage
\section{Number System}
\subsection{Binary}
\begin{tcolorbox}[colback=black!3!white,colframe=black!60!white,title=\begin{defn}Binary number system \label{Binary number system}\end{defn}]
The binary number system uses the digits $0,1$ to express itself. In particular the positive integers are represented as:
\begin{align}
\sum_{i=0}^{n} a2^{i}
\end{align}
where $a \in \mathbb{B}$ and $\mathbb{B} = \{0,1\}$. Different number systems are usually expressed with subscripts. E.g. $100101_{two}$. 
\end{tcolorbox}
\subsection{Converting to base $n$}
We can utilise the division algorithm to achieve this. That is, for some base $n$ to convert from  base $10$ we divide by $n$ to get remainders.
\begin{tcolorbox}[colback=black!3!white,colframe=black!60!white,title=\begin{exmp}Division of binary \label{Division of binary}\end{exmp}]
        
                \begin{align}
                19 \div 2 = 9 R 1 \\
		9 \div 2 = 4 R 1 \\
		4 \div 2 = 2 R 0 \\
		2 \div 2 = 1 R 0 \\
		1 \div 2 = 0 R 1 
                \end{align}
\end{tcolorbox}
\subsection{The division algorithm}
\begin{tcolorbox}[colback=black!3!white,colframe=black!60!white,title=\begin{thm}The division algorithm \label{The division algorithm}\end{thm}]
	Given any integers $a,b \in \mathbb{Z}$ and $b \neq 0$, there are unique integers $q,r \in \mathbb{Z}$ such that $a = qb+r$ and $0 \le r < |b|$.
\end{tcolorbox}
\subsection{The Euclidean algorithm}
The euclidean algorithm utilises the division algorithm to find $gcd(m,n)=b$ where $m,n,b \in \mathbb{Z}$.
\begin{tcolorbox}[colback=black!3!white,colframe=black!60!white,title=\begin{defn}Greatest Common Divisor \label{Greatest Common Divisor}\end{defn}]
The greatest common divisors of two numbers $m,n$ where $m,n \in \mathbb{Z}$ is the greatest number $\zeta$ such that $\zeta  \mid  m$ and $\zeta  \mid  n$. It is denoted as $gcd(m,n)$.
\end{tcolorbox}
Then, through division, observe that 
$n = mb + r$
In particular, the key observation would be $gcd(r,m) = gcd(n,m) = b$.
Repeat this process until one of the numbers reaches $0$.
\subsection{Modular Arithmetic}
Modular arithmetic ensures that the two numbers have the same remainder.
\begin{tcolorbox}[colback=black!3!white,colframe=black!60!white,title=\begin{exmp}Modular Arithmetic \label{Modular Arithmetic}\end{exmp}]
        The numbers $19$ and $21$ are congruent to modulo $2$. That is, they both have remainder $1$.
                \begin{align}
                19   \equiv 21 \bmod 2	 
                \end{align}
\end{tcolorbox}
The notation $a \bmod n $ can also be used as a notation to denote the remainder of the integer $a$. Furthermore, the modular arithmetic can be subtracted, added and multiplied as usual. In particular,
\begin{tcolorbox}[colback=black!3!white,colframe=black!60!white,title=\begin{exmp}Properties of Modular Arithmetic \label{Properties of Modular Arithmetic}\end{exmp}]
        Consider the following examples
                \begin{align}
                x   \equiv 3 \bmod n \\
		y   \equiv 5 \bmod n \\
		x+y   \equiv 8 \bmod n \\
		x \times y   \equiv 15 \bmod n 
                \end{align}
\end{tcolorbox}
Because of these properties, indeed
\begin{tcolorbox}[colback=black!3!white,colframe=black!60!white,title=\begin{cor}Power of modular arithmetic \label{Power of modular arithmetic}\end{cor}]
        Then, from multiplication property, the following holds true. If for some integers $x,y$ the property $x   \equiv y \bmod n $ holds, then
                \begin{align}
                x^{k}   \equiv y^{k} \bmod n 
                \end{align}
		also holds.
\end{tcolorbox}
\subsection{Use of Modular Arithmetic}
Let $N$ represent the number of bits used in a system. Then, there are $2^{N}$ bits of string length $N$ can be used to represent the numbers in the integer range $[-2^{N-1},2^{N-1}-1]$. In modular arithmetic, the integer $x$ can be then represented as $x \bmod 2^{N}$. This is two's complement. For example,
\begin{tcolorbox}[colback=black!3!white,colframe=black!60!white,title=\begin{exmp}Two's Complement \label{}\end{exmp}]
        
\begin{table}[H]
	\centering
	\caption{Two's Complement}
	\label{tab:two}
	\begin{tabular}{|c|c|c|c|c|c|}
		\hline
	$x$ & $x \bmod 2^{3}$ & String & $x$ & $x \bmod 2^{3}$ & String  \\
	\hline
	$0$ & $0$ & $000$ & $-4$ &$4$ & $100$ \\
	$1$ & $ 1$ & $ 001$ & $-3$ & $5$ & $101$ \\
	$2$ & $2$ & $010$ & $-2$ & $6$ & $110$ \\
	$3$ & $3$ & $ 011$ & $-1$ & $7$& $111$ \\
	\hline
	\end{tabular}
\end{table}
\end{tcolorbox}
\subsection{Real Numbers}
Real numbers consist of every possible numbers that are not complex. There are an infinite amount of real numbers in the interval $[0,1]$. See CS $130$ for this.
\subsection{Rational Numbers}
A rational number has the form $\frac{m}{n}$ where $m,n \in \mathbb{Z}$ and $n \neq 0$. We can always choose $ m$ and $n$ s.t. $n \ge 1$ and $gd(m,n)=1$.
\subsection{Irrational Numbers}
An algebraic number is a real number such as  $\sqrt{2}$ and $-\sqrt{2}  $. It is a solution of a polynomial equation with rational coefficients
\begin{tcolorbox}[colback=black!3!white,colframe=black!60!white,title=\begin{defn}Transcendental numbers \label{Transcendental numbers}\end{defn}]
Transcendental numbers are real numbers which cannot be solutions of polynomial equations with rational coefficients. Examples include $\pi$ and $e$.
\end{tcolorbox}
\section{Axioms}
\subsection{Algebraic Axioms}
\begin{tcolorbox}[colback=black!3!white,colframe=black!60!white,title=\begin{axiom}Commutativity \label{Commutativity}\end{axiom}]
        It follows that
                \begin{align}
                x+y = y+x \land x \times  y = y \times x
                \end{align}
\end{tcolorbox}
\begin{tcolorbox}[colback=black!3!white,colframe=black!60!white,title=\begin{axiom}Associativity \label{Associativity}\end{axiom}]
        It follows that
                \begin{align}
			x+(y+z) = (x+y)+z \land x \times  (y \times z ) = (x \times y) \times z
                \end{align}
\end{tcolorbox}
\begin{tcolorbox}[colback=black!3!white,colframe=black!60!white,title=\begin{axiom}Distrubitivity of $\times$ over + \label{Distrubitivity of * over +}\end{axiom}]
It follows that
                \begin{align}
                x \times  ( y+z) = x \times y +x \times z
                \end{align}
\end{tcolorbox}
\begin{tcolorbox}[colback=black!3!white,colframe=black!60!white,title=\begin{axiom}Additive Identity \label{Additive Identity}\end{axiom}]
        $\exists x . y+x=y$
                \begin{align}
                \text{In particular, }x=0
                \end{align}
\end{tcolorbox}
\begin{tcolorbox}[colback=black!3!white,colframe=black!60!white,title=\begin{axiom}Multiplicative Identity \label{Multiplicative Identity}\end{axiom}]
        $\exists x . yx=y$
                \begin{align}
                \text{In particular, } $x=1$		
                \end{align}
\end{tcolorbox}
\begin{tcolorbox}[colback=black!3!white,colframe=black!60!white,title=\begin{axiom}Distinction \label{Distinction}\end{axiom}]
        Multiplicative and additive identities are distinct. That is,
                \begin{align}
                1 \neq 0
                \end{align}
\end{tcolorbox}
So far, all the above axioms hold for $\mathbb{N}$. However, once we add the following axiom:
\begin{tcolorbox}[colback=black!3!white,colframe=black!60!white,title=\begin{axiom}Additive Inverse \label{Additive Inverse}\end{axiom}]
                \begin{align}
                
        \exists -x.x+(-x)=0
                \end{align}
\end{tcolorbox}
So far, all above axioms hold for $\mathbb{Z}$. However, once we add the following axiom:
\begin{tcolorbox}[colback=black!3!white,colframe=black!60!white,title=\begin{axiom}Multiplicative Inverse \label{Multiplicative Inverse}\end{axiom}]
        If $x \neq 0$, then $\exists x^{-1}.x \times x^{-1}=1$
\end{tcolorbox}
\subsection{Ordering Axioms}
\begin{tcolorbox}[colback=black!3!white,colframe=black!60!white,title=\begin{axiom}Transitivity of ordering \label{Transitivity of ordering}\end{axiom}]
                \begin{align}
        x<y \land y<z \implies x<z
                \end{align}
\end{tcolorbox}
\begin{tcolorbox}[colback=black!3!white,colframe=black!60!white,title=\begin{axiom}The trichotomy law \label{The trichotomy law}\end{axiom}]
        Exactly one of the following is true:
                \begin{align}
                x<y \lor y<x \lor x=y
                \end{align}
\end{tcolorbox}
\begin{tcolorbox}[colback=black!3!white,colframe=black!60!white,title=\begin{axiom}Preservation of ordering under addition \label{Preservation of ordering under addition}\end{axiom}]
        If $x<y$, then
                \begin{align}
                x+z < y +z
                \end{align}
\end{tcolorbox}
\begin{tcolorbox}[colback=black!3!white,colframe=black!60!white,title=\begin{axiom}Preservation of ordering under multiplication \label{Preservation of ordering under multiplication}\end{axiom}]
        If $0<z$ and $x<y$ then 
                \begin{align}
                x \times z < y \times z
                \end{align}
\end{tcolorbox}
So far, all the above axioms hold for $\mathbb{Q}$. However, once we add the following axiom:
\begin{tcolorbox}[colback=black!3!white,colframe=black!60!white,title=\begin{axiom}Completeness \label{Completeness}\end{axiom}]
        Every non-empty subset that is bounded above has a least upper bound.
\end{tcolorbox}
\subsection{Ordering}
\begin{tcolorbox}[colback=black!3!white,colframe=black!60!white,title=\begin{defn}Upper bound \label{Upper bound}\end{defn}]
A real number $u$ is an upper bound of $S$ if $u \ge x \; \forall x \in S$
\end{tcolorbox}
\begin{tcolorbox}[colback=black!3!white,colframe=black!60!white,title=\begin{defn}Lower bound \label{Lower bound}\end{defn}]
A real number $u$ is a lower bound of $S$ if $l \ge x \; \forall x \in S$
\end{tcolorbox}
\begin{tcolorbox}[colback=black!3!white,colframe=black!60!white,title=\begin{defn}Supremum \label{Supremum}\end{defn}]
A real number $U$ is supremum of $S$ if $ U$ is an upper bound of $S$ and $U\le u$ for every upper bound $u$ of $S$. That is, it is the first upper bound.
\end{tcolorbox}
\begin{tcolorbox}[colback=black!3!white,colframe=black!60!white,title=\begin{defn}Infimum \label{Infimum}\end{defn}]
A real number $L$ is the infimum of $S$ if $L$ is a lower bound of $S$ and $L \ge l$ for every lower bound $l$ of $S$. That is, it is the first lower bound.
\end{tcolorbox}
\subsection{Archimedes Property of Real}
\begin{tcolorbox}[colback=black!3!white,colframe=black!60!white,title=\begin{thm}Archimedes Property of Reals \label{Archimedes Property of Reals}\end{thm}]
	Given any $\varepsilon \in \mathbb{R}^{+}$, $\exists n \in \mathbb{N} . n \varepsilon >1$ 
	\begin{proof}
		Assume $n \varepsilon \le 1$. Then,
		\begin{align}
			&\forall n \text{ that } \{ n \varepsilon | n \in \mathbb{N}\} \text{ has an upper bound} \\
			&\text{By completeness it has a least upper bound $l$} \\
			\implies& \forall n, n \varepsilon \le l \\
			\implies& (n+1) \varepsilon \le l \\
			\iff &(n+1)\varepsilon - \varepsilon \le l - \varepsilon\\
			\iff &n \varepsilon \le  l - \varepsilon \\
			\implies &l - \varepsilon \text{ is also an upper bound}
		\end{align}
		However, this is a contradiction since we already assumed that $l$ is the least upper bound when clearly $ l - \varepsilon < l$
	\end{proof}
\end{tcolorbox}
\end{document}
