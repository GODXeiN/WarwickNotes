\documentclass[a4paper]{article}

\usepackage[utf8]{inputenc}
\usepackage[T1]{fontenc}
\usepackage{textcomp}
\usepackage{listings}
\usepackage{lmodern}
\usepackage{amsfonts}
\usepackage{titling}
\usepackage{lipsum}
\usepackage[left=1in, right=1in, bottom=1in, top=1in]{geometry}
\usepackage{amsthm}
\usepackage{tcolorbox}
\usepackage{hyperref}
\usepackage{xcolor}
\usepackage{graphicx}
\usepackage{makeidx}
\usepackage{tikz}
\usepackage{cases}
\usepackage{apacite}
\usepackage{tkz-berge}
\usepackage{url}
\usepackage{tgtermes}
\usepackage{sectsty}
\usepackage{subcaption}
\usepackage{setspace}
\usepackage{float}
\usepackage{amsmath, amssymb}


% figure support
\usepackage{import}
\usepackage{xifthen}
\pdfminorversion=7
\usepackage{pdfpages}
\usepackage{transparent}
\newcommand{\incfig}[2][1]{%
    \def\svgwidth{#1\columnwidth}
    \import{./figures/}{#2.pdf_tex}
}

%mathstyling
\theoremstyle{plain}
\newtheorem{thm}{Theorem}[section]
\newtheorem{lem}[thm]{Lemma}
\newtheorem{prop}[thm]{Proposition}
\newtheorem*{cor}{Corollary}

\theoremstyle{definition}
\newtheorem{defn}{Definition}[section]
\newtheorem{conj}{Conjecture}[section]
\newtheorem{exmp}{Example}[section]
\newtheorem{axiom}{Axiom}
\theoremstyle{remark}
\newtheorem*{rem}{Remark}
\newtheorem*{note}{Note}

\pdfsuppresswarningpagegroup=1

\begin{document}
	\begin{titlepage}
	\begin{center}
	\large
	University of Warwick \\
	Department of Computer Science \\
	\huge
	\vspace{50mm}
	\rule{\linewidth}{0.5pt} \\
	CS131 \\
	\vspace{5mm}
	\Large
	Mathematics for Computer Scientists II
	\rule{\linewidth}{0.5pt}
	\vspace{5mm}
	\begin{figure}[H]
	\centering
	\includegraphics[width=0.4\textwidth]{crest_black.eps}
	\end{figure}
	\vspace{37mm}
	Cem Yilmaz \\
	\today
	\end{center}
	\end{titlepage}
	\newpage
	
\section{Number System}
\subsection{Binary}
\begin{tcolorbox}[colback=black!3!white,colframe=black!60!white,title=\begin{defn}Binary number system \label{Binary number system}\end{defn}]
The binary number system uses the digits $0,1$ to express itself. In particular the positive integers are represented as:
\begin{align}
\sum_{i=0}^{n} a2^{i}
\end{align}
where $a \in \mathbb{B}$ and $\mathbb{B} = \{0,1\}$. Different number systems are usually expressed with subscripts. E.g. $100101_{two}$. 
\end{tcolorbox}
\subsection{Converting to base $n$}
We can utilise the division algorithm to achieve this. That is, for some base $n$ to convert from  base $10$ we divide by $n$ to get remainders.
\begin{tcolorbox}[colback=black!3!white,colframe=black!60!white,title=\begin{exmp}Division of binary \label{Division of binary}\end{exmp}]
        
                \begin{align}
                19 \div 2 = 9 R 1 \\
		9 \div 2 = 4 R 1 \\
		4 \div 2 = 2 R 0 \\
		2 \div 2 = 1 R 0 \\
		1 \div 2 = 0 R 1 
                \end{align}
\end{tcolorbox}
\subsection{The division algorithm}
\begin{tcolorbox}[colback=black!3!white,colframe=black!60!white,title=\begin{thm}The division algorithm \label{The division algorithm}\end{thm}]
	Given any integers $a,b \in \mathbb{Z}$ and $b \neq 0$, there are unique integers $q,r \in \mathbb{Z}$ such that $a = qb+r$ and $0 \le r < |b|$.
\end{tcolorbox}
\subsection{The Euclidean algorithm}
The euclidean algorithm utilises the division algorithm to find $gcd(m,n)=b$ where $m,n,b \in \mathbb{Z}$.
\begin{tcolorbox}[colback=black!3!white,colframe=black!60!white,title=\begin{defn}Greatest Common Divisor \label{Greatest Common Divisor}\end{defn}]
The greatest common divisors of two numbers $m,n$ where $m,n \in \mathbb{Z}$ is the greatest number $\zeta$ such that $\zeta  \mid  m$ and $\zeta  \mid  n$. It is denoted as $gcd(m,n)$.
\end{tcolorbox}
Then, through division, observe that 
$n = mb + r$
In particular, the key observation would be $gcd(r,m) = gcd(n,m) = b$.
Repeat this process until one of the numbers reaches $0$.
\end{document}
